\documentclass[12pt]{article}

\usepackage[utf8]{inputenc}
\usepackage{newunicodechar}
\newunicodechar{ℝ}{\mathbb{R}}
\usepackage{EngReport}
\usepackage{listings}
\usepackage{cancel}
\usepackage{comment}
\usepackage{amssymb}
\usepackage{amsthm}
\usepackage{amsmath}
\usepackage{graphicx}
\usepackage{setspace}
\usepackage{geometry}
\usepackage{xcolor}  % Required for coloring in listings

\graphicspath{{Images/}}
\onehalfspacing
\geometry{letterpaper, portrait, includeheadfoot=true, hmargin=1in, vmargin=1in}

% Define custom colors
\definecolor{myblue}{RGB}{0, 128, 255}
\definecolor{mygreen}{RGB}{34, 139, 34}
\definecolor{myorange}{RGB}{255, 140, 0}
\definecolor{mygray}{RGB}{128, 128, 128}
\definecolor{mypurple}{RGB}{148, 0, 211}
\definecolor{myred}{RGB}{255, 69, 0}

% Configure listings for Python with custom styles
\lstset{
    language=Python,             % Set language to Python
    basicstyle=\ttfamily\small,  % Use a smaller monospace font
    keywordstyle=\color{myblue}\bfseries,  % Keywords in blue and bold
    commentstyle=\color{mygreen}\itshape,  % Comments in green and italic
    stringstyle=\color{myorange},          % Strings in orange
    numberstyle=\color{mygray},            % Line numbers in gray
    identifierstyle=\color{mypurple},      % Functions and variables in purple
    morekeywords={print, len, range},      % Define additional Python keywords
    showstringspaces=false,                % Do not show spaces in strings
    breaklines=true,                       % Enable line breaking
    numbers=left,                          % Add line numbers to the left
    numbersep=5pt,                         % Space between line numbers and code
    frame=single,                          % Add a box around the code
    rulecolor=\color{mygray},              % Frame color
    moredelim=[is][\color{myred}]{@@}{@@}, % Custom inline LaTeX coloring
}

\begin{document}
\renewcommand{\familydefault}{\rmdefault}

\begin{titlepage}
    \null % This is a TeX command that does nothing but is necessary for vfill to work correctly
    \vfill
    \begin{center}
        {\fontsize{40}{48}\selectfont \bfseries CSC236 Exam Review}
        \vspace{20pt} \\
        {\LARGE Notes from CSC236 Lecture 12} \\
        \vspace{20pt}
        \textbf{Alexander He Meng}
        \vspace{8pt}
        \\ Typed on November 27, 2024
    \end{center}
    \vfill
\end{titlepage}

\pagestyle{fancy}
\fancyhf{}
\setlength{\headheight}{30pt}
\renewcommand{\headrulewidth}{0.4pt}
\renewcommand{\footrulewidth}{0.4pt}
\lhead{\large \textbf{CSC236 UTM} \\ \textbf{Mock Exam }\scriptsize(thx ethan!)\normalsize \textbf{ Solutions}}
\rhead{\large \textbf{Fall 2024} \\ \textbf{Prepared for Dec 17}}
\rfoot{\textbf{Page \thepage}}
\lfoot{}
\pagebreak
\normalsize

\section*{Question \#1}
Consider a program that takes an array of intervals \texttt{intervals} where \texttt{intervals[i]} = \( [\texttt{start}_i, \texttt{end}_i] \) and returns an optimal schedule: \\
\begin{lstlisting}
def optimalschedule(intervals):
    sort intervals by the end times
    S = []
    f = -infty
    for i in [1, ..., n]:
        if start_i >= f:
            S.append([start_i, end_i])
            f = end_i
    return S
\end{lstlisting}
\leavevmode\\
\underline{Definitions, Notes, and Examples:}
\begin{itemize}
    \item An \textbf{optimal schedule} is a subarray of \texttt{intervals} in which all the intervals are non-overlapping, and the subarray has the maximum possible size.
    \item \( [1, 2] \) and \( [2, 3] \) are non-overlapping.
    \item There may be multiple optimal schedules for an arbitrary array of intervals.
    \item All optimal schedules have the same size.
    \item In general, \( \texttt{intervals} = [[\texttt{start}_1, \texttt{end}_1], \dots, [\texttt{start}_n, \texttt{end}_n]] \) for some \( n \in \mathbb{N}^+ \) and \( \texttt{start}_i, \texttt{end}_i \in \mathbb{R}^+ \).
    \item The length of \texttt{intervals} is at least $1$ (\texttt{intervals} is non-empty).
    \item If \( S \) is the subarray (in the program) at the \( j^{\text{th}} \) iteration and there exists some optimal schedule \( Opt \) such that \( [\texttt{start}_i, \texttt{end}_i] \in Opt \iff [\texttt{start}_i, \texttt{end}_i] \in S \), then \( S \) is \textbf{looking good}.
    \item Let \( S \) be the subarray on the \( j^{\text{th}} \) iteration of the program. Define the predicate, \( P(S): S \) is \textbf{looking good}.
\end{itemize}
\leavevmode\\
\underline{\textbf{Claim:}} The \texttt{optimalschedule()} program terminates.
\begin{proof}
\leavevmode\\
    Consider the loop variant \( Var = n - i \) in the \( i^{\text{th}} \) iteration. Denote \( \widetilde{Var} \) as the loop variant in the subsequent (\( (i + 1)^{\text{th}} \)) iteration. \\
    \\
    Then, notice that \( \widetilde{Var} = n - (i + 1) < n - i = Var \), where \( n, i \in \mathbb{N} \) and \( i \leq n \); \( (n - i) \in \mathbb{N} \). \\
    \\
    Therefore, the loop variant decreases in every subsequent iteration. With \( n \) iterations and a step to return the result, the program terminates after \( n + 1 \) iterations. \\ 
\end{proof}
\leavevmode\\
\underline{\textbf{Claim:}} \( S \) is \textbf{looking good} at the beginning of the first iteration.
\begin{proof}
\leavevmode\\
    At the start of the first iteration, \( S = \texttt{[]} \), which is trivially a subset of every optimal schedule \( Opt \). Indeed, \( S \) satisfies the definition of \textbf{looking good}. \\
    \\
    Namely, there are no positive integers \( i < j = 1 \), making \( P(n) \) trivially hold. \\
\end{proof}
\leavevmode\\
\underline{\textbf{Claim:}} If \( S \) is \textbf{looking good} at the beginning of the first iteration, then the first iteration executes, and \( S \) is looking \textbf{looking good} at the beginning of the second iteration.
\begin{proof}
\leavevmode\\
    Assume \( S \) is \textbf{looking good} at the beginning of the first iteration. \\
    \\
    Since \( i = 1 \), the first iteration of the loop executes. \\
    As well, the if-statement on \textit{Line 6} of the program evaluates to \texttt{true} as \( \texttt{f} = \texttt{-infty} \). \\
    \\
    Then, \textit{Line 7} appends \( [\texttt{start}_1, \texttt{end}_1] \) to \( S \) and \textit{Line 8} updates \texttt{f} to become \( \texttt{end}_i \). \\
    \\
    This completes the first iteration. \\
    \\
    For the second iteration of the loop, consider an arbitrary optimal schedule \( Opt \). Construct a new schedule \( Opt' \) (of the same size) obtained by replacing the first interval with \( [\texttt{start}_1, \texttt{end}_1] \). \\
    \\
    The \texttt{intervals} list is sorted, so \( Opt' \) is a schedule with no overlaps. Namely, \( \texttt{end}_1 \leq \texttt{end}_a \), where \( \texttt{end}_a \) is the first endpoint of \( Opt \). \\
    \\
    \( Opt' \) agrees with \( S \) on the first \( j - 1 = 1 \) interval. Thus, \( Opt' \) must be an optimal schedule. \\
    \\
    Therefore, \( S \) is \textbf{looking good} at the start of the second iteration. \\
\end{proof}
\leavevmode\\
\underline{\textbf{Claim:}} If \( S \) is \textbf{looking good} at the beginning of every iteration, including the iteration after the last that fails to be executed, then \( S \) is an optimal schedule.
\begin{proof}
\leavevmode\\
    Assume \( S \) is \textbf{looking good} at the beginning of every iteration, including the iteration after the last that fails to be executed. Then, there exists an optimal schedule \( Opt \) such that for all \( i < n + 1 \) (\( n \) is the last iteration number),
    \[
        [\texttt{start}_i, \texttt{end}_i] \in Opt \iff [\texttt{start}_i, \texttt{end}_i] \in S \text{.}
    \]
    By definition, \( Opt \) is a maximal size subarray of intervals that are non-overlapping. Notice that \( S \) is constructed to be in the same way, through greedily selecting intervals based on the nearest start time (\textit{Line 6} of the program). Thus, \( Opt \) and \( S \) are subarrays of the same size. \\
    \\
    Since all intervals in \( Opt \) and \( S \) are in common, it follows that \( S \) is equivalent to \( Opt \). This makes \( S \) an optimal schedule. \\
\end{proof}
\leavevmode\\
\underline{\textbf{Claim:}} If \( S \) is \textbf{looking good} at the beginning of the \( j^{\text{th}} \) iteration implies \( S \) is \textbf{looking good} at the beginning of the \( (j + 1)^{\text{th}} \) iteration, then \texttt{optimalschedule()} is correct.
\begin{proof}
\leavevmode\\
    Denote the loop invariant:
    \[
        Q(k): P(s) \text{ holds at the beginning of the \( k ^{\text{th}} \) iteration.}
    \]
    To show that \texttt{optimalschedule()} is correct—that is, \texttt{optimalschedule()} returns an optimal schedule, show that \( Q(k) \) is true for all \( k \in \mathbb{N} \). \\
    \\
    Proceed using the principle of simple induction on \( Q(k) \) over \( k \in \mathbb{N} \). \\
    \\
    \underline{Base Case:} \\
    Let \( i = 0 \). \\
    The claim that \( S \) is \textbf{looking good} at the beginning of the first iteration has been proved above. \\
    \\
    \underline{Induction Hypothesis:} \\
    Assume for some \( k \in \mathbb{N} \), \( Q(k) \) is holds. \\
    \\
    This means \( S \) is \textbf{looking good} at the beginning of the \( k^{\text{th}} \) iteration. \\
    \\
    \underline{Induction Step:} \\
    By the induction hypothesis and the assumption, \( S \) is also \textbf{looking good} at the beginning of the \( (j + 1)^{\text{th}} \) iteration. \\
    \\
    \underline{Induction Conclusion:} \\
    Therefore, \( Q(k) \) holds for all \( k \in \mathbb{N} \). \\
    \\
    Next, the claim that the \texttt{optimalschedule()} program terminates has also already been proved. Namely, the program's loop terminates at the beginning of the \( (n + 1)^{\text{th}} \) iteration. \\
    \\
    The last claim proved guarantees that if \( S \) is \textbf{looking good} at the beginning of every iteration, including the iteration after the last that fails to be executed, then \( S \) is an optimal schedule. This means \( S \) is constructed to be an optimal schedule after the program's loop terminates. \\
    \\
    By finally returning \( S \), the program satisfies its postcondition of returning an optimal schedule. Therefore, the program is correct. \\
\end{proof}
\pagebreak

\section*{Question \#2}
Prove that \( f(n) = \lceil \sqrt(n) \rceil - \lfloor \sqrt(n) - 4 \rfloor\) is asymptotically constant (i.e. \( \Theta(1)\)).
\begin{proof}
\leavevmode\\
    By definition, if \(x\) and \(y\) are arbitrary real numbers, then \[ (x \leq \lceil x \rceil < x + 1) \] and \[ (y - 1 < \lfloor y \rfloor \leq y) \text{.} \] \\
    Rewrite the second inequality as \( -y \leq - \lfloor y \rfloor < - (y - 1) \). \\
    \\
    By adding the two inequalities, it follows that \( x - y \leq \lceil x \rceil - \lfloor y \rfloor < x + 1 - (y - 1) = x - y + 2 \). \\
    \\
    Let \( x = \sqrt{n} \) and \( y = \sqrt{n} - 4 \), for arbitrary natural \( n \). \\
    Then, \( \lceil x \rceil - \lfloor y \rfloor = \lceil \sqrt{n} \rceil - \lfloor \sqrt{n} - 4 \rfloor = f(n) \). As well, \( x - y = \cancel{\sqrt{n}} - (\cancel{\sqrt{n}} - 4) = 4 \). \\
    \\
    This means \( x - y \leq \lceil x \rceil - \lfloor y \rfloor < x - y + 2 \implies 4 \leq f(n) < 4 + 2 \implies 4 \leq f(n) < 6 \). \\
    \\
    Let \( n_0 = 0, c = 4, d = 6 \). Let \( g(n) = 1 \). \\
    Notice that \( 4 \leq f(n) < 6 \implies cg(n) \leq f(n) \leq dg(n) \), for all \( n \geq n_0 = 0 \) with \( c = 4, d = 6 \). \\
    \\
    Therefore, \( f(n) \in \Theta(g(n)) \implies f(n) \in \Theta(1) \). Indeed, \( f(n) \) is asymptotically constant. \\
\end{proof}
\pagebreak

\section*{Steps to show that a DFA does not accept a language}
\begin{enumerate}
    \item Show that there exists \( x, y \in \Sigma ^* \) such that \( \hat{\delta}(q_0, x) = \hat{\delta}(q_0, y) \).
    \item Show that there exists \( z \in \Sigma^* \) such that \( xz \in L \iff yz \notin L \).
    \item Clarify the contradiction that \( \hat{\delta}(q_0, xz) = \hat{\delta}(q_0, yz) \implies \hat{\delta}(q_0, xz) \in L \).
\end{enumerate}
\pagebreak

\section*{Question \#3}
Prove that \( L = \{ a^{n^2} \mid n \in \mathbb{N} \} \) is \textbf{not} a regular language.
\begin{proof}
\leavevmode\\
    Assume, for contradiction, that \( L = \{ a^{n^2} \mid n \in \mathbb{N} \} \) is regular. Then there exists a deterministic finite automata (DFA) \( \mathcal{D} = \{ Q, \Sigma, \delta, s, F \} \) that accepts \( L \). \\
    Let \( |Q| = k \), where \( k \) is the number of states in \( \mathcal{D} \). \\
    \\
    Since \( L \) contains strings of the form \( a^{n^2} \), choose \( w = a^{j^2} \), where \( j \) is large enough such that \( j^2 > k \). Clearly \( w \in L \). \\
    \\
    As the DFA processes \( w \), which has \( j^2 > k \) symbols, it must visit more states then there are in \( Q \). By the Pigeonhole Principle, at least one state must repeat. \\
    Namely, while processing \( w \), there exist integers \( \alpha \) and \( \beta \) such that:
    \[ \hat{\delta}(s, a^{\alpha}) = \hat{\delta}(s, a^{\alpha + \beta}) \text{,} \]
    where \( \beta \geq 1 \). \\
    This means that after reading the first \( \alpha \) symbols, the DFA enters some state \( q \), and reading \( \beta \) additional symbols loops back to \( q \). \\
    \\
    Because \( \mathcal{D} \) accepts \( w = a^{j^2} \), it follows that:
    \[
        \hat{\delta}(s, a^{j^2}) \in L \text{.}
    \]
    Now consider the strings \( a^{j^2 + \beta} \), \( a^{j^2 + 2\beta} \), and so on. Since the DFA loops at state \( q \), adding multiples of \( \beta \) symbols to \( w \) does not change the final state. \\
    Therefore:
    \[
        \hat{\delta}(s, a^{j^2 + \beta}) \in L \;\;\; \text{and} \;\;\; \hat{\delta}(s, a^{j^2 + 2\beta}) \in L
    \]
    Thus, the DFA also accepts these strings. \\
    \\
    Recall that with \( k \) states processing the first \( k + 1 \) symbols of \( w \) must cause a state to repeat (the Pigeonhole Principle).
    \begin{itemize}
        \item Let \( \alpha \) be the number of symbols leading up to the first occurrence of a repeated state \( q \).
        \item Let \( \beta \) be the number of states causing the DFA to loop back to \( q \).
        \item Let \( \gamma \) account for any remaining symbols to reach the end of \( w = a^{j^2} \).
    \end{itemize}
    Thus, \( j^2 = \alpha + \beta + \gamma \). Note that \( \hat{\delta}(s, a^{\alpha}) = \hat{\delta}(q, a^{\beta}) = q \); denote this as \textit{Corollary 1}. \\
    \\
    It is now possible to show explicitly the strings which the DFA accepts. \\
    Consider that:
    \begin{equation*}
        \begin{aligned}
            \hat{\delta}(s, a^{j^2 + \beta}) &= \hat{\delta}(s, a^{(\alpha + \beta + \gamma) + \beta}) \\
            &= \hat{\delta}(\hat{\delta}(s, a^{\alpha}), a^{\beta + \beta + \gamma}) \\
            &= \hat{\delta}(q, a^{\beta + \beta + \gamma}) \text{, by \textit{Corollary 1}} \\
            &= \hat{\delta}(\hat{\delta}(q, a^{\beta}), a^{\beta + \gamma}) \\
            &= \hat{\delta}(q, a^{\beta + \gamma}) \text{, by \textit{Corollary 1}} \\
            &= \hat{\delta}(\hat{\delta}(s, a^{\alpha}), a^{\beta + \gamma}) \text{, by \textit{Corollary 1}} \\
            &= \hat{\delta}(s, a^{\alpha + \beta + \gamma}) \\
            &= \hat{\delta}(s, a^{j^2})
        \end{aligned}
    \end{equation*}
    \\
    This equivalence shows that \( \mathcal{D} \) accepts \( w = a^{j^2 + \beta} \). By continually applying \textit{Corollary 1} in the same argument, \( \mathcal{D} \) also accepts \( a^{j^2 + 2\beta}, a^{j^2 + 3\beta} \). \\
    \\
    However, notice that one of \( a^{j^2 + 2\beta}, a^{j^2 + 3\beta} \) is not a square and, thus, not a member of \( L \). Yet, \( \mathcal{D}  \) accepts both strings. This is a contradiction. \\
    \\
    Therefore, \( L = \{ a^{n^2} \mid n \in \mathbb{N} \} \) must not be regular. \\
\end{proof}
\leavevmode\\
\textit{Here's an alternative proof using the \textbf{Pumping Lemma}.}
\begin{proof}
\leavevmode\\
    This proof demonstrates that \( L = \{ a^{n^2} \mid n \in \mathbb{N} \} \) is not a regular language using the pumping lemma for regular languages. \\
    \\
    The pumping lemma states that if \( L \) is a regular language, then there exists a pumping length \( p \geq 1 \) such that for all \( w \in L \) where \( |w| \geq p \), \( w \) can be written as \( w = xyz \mid_{x, y, z \in \Sigma^*} \) satisfying:
    \[
    |xy| \leq p, \quad |y| \geq 1, \quad \text{and} \quad xy^iz \in L \text{, for all \( i \in \mathbb{N} \).}
    \]
    \\
    Assume for contradiction that \( L \) is regular. Let \( p \geq 1 \) be the pumping length given by the pumping lemma. \\
    Choose \( w = a^{p^2} \in L \). Notice that \( |w| = p^2 \geq p \), so the conditions of the pumping lemma hold. \\
    \\
    By the pumping lemma, \( w \) can be split into \( w = xyz \) such that:
    \begin{itemize}
        \item \( |xy| \leq p \),
        \item \( |y| \geq 1 \),
        \item \( xy^iz \in L \), for all \( i \in \mathbb{N} \).
    \end{itemize}
    Since \( |xy| \leq p \), the string \( xy \) consists of at most \( p \) \( a \)'s. Still, \( y \) consists entirely of \( a \)'s, so write \( y = a^k \) for some \( k \geq 1 \). \\
    \\
    Now, consider \( i = 2 \). The pumped string \( xy^2z \) is:
    \[
    xy^2z = x a^{2k} z.
    \]
    The length of \( xy^2z \) is:
    \[
    |xy^2z| = |x| + 2|y| + |z| = (|x| + |y| + |z|) + |y| = p^2 + k.
    \]

    \noindent To remain in \( L \), the length \( p^2 + k \) must be a perfect square. However, there are specific values leading to a contradiction. Let \( p = 2 \), so \( p^2 = 4 \). Then:
    \[
    w = a^4 \quad \text{and} \quad y = a^1 \, (\text{since \( |y| \geq 1 \)}).
    \]
    Pumping \( y \) with \( i = 2 \), it follows that:
    \[
    xy^2z = a^{4 + 1} = a^5.
    \]
    The string \( a^5 \) is not in \( L \), because \( 5 \) is not a perfect square. \\
    \\
    This contradicts the pumping lemma, which requires \( xy^iz \in L \) for all \( i \geq 0 \). \\
    \\
    Therefore, \( L \) is not a regular language. \\
\end{proof}
\pagebreak

\section*{Question \#4}
Prove that \( L = \{ 0^n1^n \mid n \in \mathbb{N} \} \) is \textbf{not} a regular language.\begin{proof}
\leavevmode\\
    Seeking a contradiction, assume that \( L \) is a regular language. Then, by the definition of regular languages, there exists a deterministic finite automata (DFA) \( M \) with \( p \) states that accepts \( L \). \\
    \\
    Let \( n \in \mathbb{N} \) such that \( n > p \). Choose \( w = 0^{n + 300}1^{n + 300} \). \\
    Clearly, \( w \in L \), so \( M \) accepts \( w \). By the Pigeonhole Principle, since \( M \) has \( p \) states and processes \( w \), some state in \( M \) must be repeated while reading the first \( n + 300 \) zeroes of \( w \). \\
    \\
    Let \( x, y, z \in \Sigma^* \) be strings such that \( w = xyz \), where:
    \begin{itemize}
        \item \( xy \) corresponds to the prefix of \( w \) up to the repeated state,
        \item \( y \neq \varepsilon \) (i.e., \( y \) is the part of \( w \) causing the repetition),
        \item \( z \) is the remainder of \( w \).
    \end{itemize}
    Thus, \( w = 0^{n + 300}1^{n + 300} \), and \( x = 0^a \), \( y = 0^b \), \( z = 0^c1^{n+300} \), where \( a + b + c = n + 300 \) and \( b > 0 \). \\
    \\
    Now, consider the string \( w' = xy^2z \), which is obtained by repeating \( y \) once. Then:
    \[
    w' = 0^a0^{2b}0^c1^{n+300} = 0^{n + 300 + b}1^{n + 300}.
    \]
    Clearly, \( w' \notin L \) because the number of zeroes exceeds the number of ones (\( n + 300 + b > n + 300 \)). This contradicts the assumption that \( M \) accepts \( L \), as \( M \) would also accept \( w' \), which is not in \( L \). \\
    \\
    Hence, \( L \) is not a regular language. \\
\end{proof}
\pagebreak

\end{document}
