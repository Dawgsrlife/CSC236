\documentclass[12pt]{article}

\usepackage[utf8]{inputenc}
\usepackage{newunicodechar}
\newunicodechar{ℝ}{\mathbb{R}}
\usepackage{EngReport}
\usepackage{listings}
\usepackage{cancel}
\usepackage{comment}
\usepackage{amssymb}
\usepackage{amsthm}
\usepackage{amsmath}
\usepackage{graphicx}
\usepackage{setspace}
\usepackage{geometry}
\usepackage{xcolor}  % Required for coloring in listings

\graphicspath{{Images/}}
\onehalfspacing
\geometry{letterpaper, portrait, includeheadfoot=true, hmargin=1in, vmargin=1in}

% Define custom colors
\definecolor{myblue}{RGB}{0, 128, 255}
\definecolor{mygreen}{RGB}{34, 139, 34}
\definecolor{myorange}{RGB}{255, 140, 0}
\definecolor{mygray}{RGB}{128, 128, 128}
\definecolor{mypurple}{RGB}{148, 0, 211}
\definecolor{myred}{RGB}{255, 69, 0}

% Configure listings for Python with custom styles
\lstset{
    language=Python,             % Set language to Python
    basicstyle=\ttfamily\small,  % Use a smaller monospace font
    keywordstyle=\color{myblue}\bfseries,  % Keywords in blue and bold
    commentstyle=\color{mygreen}\itshape,  % Comments in green and italic
    stringstyle=\color{myorange},          % Strings in orange
    numberstyle=\color{mygray},            % Line numbers in gray
    identifierstyle=\color{mypurple},      % Functions and variables in purple
    morekeywords={print, len, range},      % Define additional Python keywords
    showstringspaces=false,                % Do not show spaces in strings
    breaklines=true,                       % Enable line breaking
    numbers=left,                          % Add line numbers to the left
    numbersep=5pt,                         % Space between line numbers and code
    frame=single,                          % Add a box around the code
    rulecolor=\color{mygray},              % Frame color
    moredelim=[is][\color{myred}]{@@}{@@}, % Custom inline LaTeX coloring
}

\begin{document}
\renewcommand{\familydefault}{\rmdefault}

\begin{titlepage}
    \null % This is a TeX command that does nothing but is necessary for vfill to work correctly
    \vfill
    \begin{center}
        {\fontsize{40}{48}\selectfont \bfseries CSC236 Exam Review}
        \vspace{20pt} \\
        {\LARGE Notes from CSC236 Lecture 12} \\
        \vspace{20pt}
        \textbf{Alexander He Meng}
        \vspace{8pt}
        \\ Typed on November 27, 2024
    \end{center}
    \vfill
\end{titlepage}

\pagestyle{fancy}
\fancyhf{}
\setlength{\headheight}{30pt}
\renewcommand{\headrulewidth}{0.4pt}
\renewcommand{\footrulewidth}{0.4pt}
\lhead{\large \textbf{CSC236 UTM} \\ \textbf{Mock Exam }\scriptsize(thx ethan!)\normalsize \textbf{ Solutions}}
\rhead{\large \textbf{Fall 2024} \\ \textbf{Prepared for Dec 17}}
\rfoot{\textbf{Page \thepage}}
\lfoot{}
\pagebreak
\normalsize

\section*{Question \#1}
Consider a program that takes an array of intervals \texttt{intervals} where \texttt{intervals[i]} = \( [\texttt{start}_i, \texttt{end}_i] \) and returns an optimal schedule: \\
\begin{lstlisting}
def optimalschedule(intervals):
    sort intervals by the end times
    S = []
    f = -infty
    for i in [1, ..., n]:
        if start_i >= f:
            S.append([start_i, end_i])
            f = end_i
    return S
\end{lstlisting}
\leavevmode\\
\underline{Definitions, Notes, and Examples:}
\begin{itemize}
    \item An \textbf{optimal schedule} is a subarray of \texttt{intervals} in which all the intervals are non-overlapping, and the subarray has the maximum possible size.
    \item \( [1, 2] \) and \( [2, 3] \) are non-overlapping.
    \item There may be multiple optimal schedules for an arbitrary array of intervals.
    \item All optimal schedules have the same size.
    \item In general, \( \texttt{intervals} = [[\texttt{start}_1, \texttt{end}_1], \dots, [\texttt{start}_n, \texttt{end}_n]] \) for some \( n \in \mathbb{N}^+ \) and \( \texttt{start}_i, \texttt{end}_i \in \mathbb{R}^+ \).
    \item The length of \texttt{intervals} is at least $1$ (\texttt{intervals} is non-empty).
    \item If \( S \) is the subarray (in the program) at the \( j^{\text{th}} \) iteration and there exists some optimal schedule \( Opt \) such that \( [\texttt{start}_i, \texttt{end}_i] \in Opt \iff [\texttt{start}_i, \texttt{end}_i] \in S \), then \( S \) is looking good.
    \item Let \( S \) be the subarray on the \( j^{\text{th}} \) iteration of the program. Define the predicate, \( P(S): S \) is looking good.
\end{itemize}
\leavevmode\\
\underline{\textbf{Claim:}} something \\
\begin{proof}
\leavevmode\\
    proofgoeshere \\
\end{proof}
\leavevmode\\
\underline{\textbf{Claim:}} something \\
\begin{proof}
\leavevmode\\
    proofgoeshere \\
\end{proof}
\leavevmode\\
\underline{\textbf{Claim:}} something \\
\begin{proof}
\leavevmode\\
    proofgoeshere \\
\end{proof}
\leavevmode\\
\underline{\textbf{Claim:}} something \\
\begin{proof}
\leavevmode\\
    proofgoeshere \\
\end{proof}
\leavevmode\\
\underline{\textbf{Claim:}} something \\
\begin{proof}
\leavevmode\\
    proofgoeshere \\
\end{proof}
\pagebreak

\section*{Question \#2}
Prove that \( f(n) = \lceil \sqrt(n) \rceil - \lfloor \sqrt(n) - 4 \rfloor\) is asymptotically constant (i.e. \( \Theta(1)\)).
\begin{proof}
\leavevmode\\
    By definition, if \(x\) and \(y\) are arbitrary real numbers, then \[ (x \leq \lceil x \rceil < x + 1) \] and \[ (y - 1 < \lfloor y \rfloor \leq y) \text{.} \] \\
    Rewrite the second inequality as \( -y \leq - \lfloor y \rfloor < - (y - 1) \). \\
    \\
    By adding the two inequalities, it follows that \( x - y \leq \lceil x \rceil - \lfloor y \rfloor < x + 1 - (y - 1) = x - y + 2 \). \\
    \\
    Let \( x = \sqrt{n} \) and \( y = \sqrt{n} - 4 \), for arbitrary natural \( n \). \\
    Then, \( \lceil x \rceil - \lfloor y \rfloor = \lceil \sqrt{n} \rceil - \lfloor \sqrt{n} - 4 \rfloor = f(n) \). As well, \( x - y = \cancel{\sqrt{n}} - (\cancel{\sqrt{n}} - 4) = 4 \). \\
    \\
    This means \( x - y \leq \lceil x \rceil - \lfloor y \rfloor < x - y + 2 \implies 4 \leq f(n) < 4 + 2 \implies 4 \leq f(n) < 6 \). \\
    \\
    Let \( n_0 = 0, c = 4, d = 6 \) and notice that \( f(n) \in \Theta(1) \). \\
\end{proof}
\pagebreak

\section*{Showing that a DFA does not accept a language}
\begin{enumerate}
    \item Show that there exists \( x, y \in \Sigma ^* \) such that \( \hat{\delta}(q_0, x) = \hat{\delta}(q_0, y) \).
    \item Show that there exists \( z \in \Sigma^* \) such that \( xz \in L \iff yz \notin L \).
    \item Clarify the contradiction that \( \hat{\delta}(q_0, xz) = \hat{\delta}(q_0, yz) \implies \hat{\delta}(q_0, xz) \in F \).
\end{enumerate}
\pagebreak

\section*{Question \#3}
Prove that \( L = \{ a^{n^2} \mid n \in \mathbb{N} \} \) is \textbf{not} a regular language.
\begin{proof}
\leavevmode\\
    Seeking a contradiction, assume there exists a DFA \( \mathcal{D} = \{ Q, \Sigma, \delta, s, F \} \) that accepts \( L = \{ a^{n^2} \mid n \in \mathbb{N} \} \). \\
    Let \( |Q| = k \). \\
    \\
    Then, choose \( w \in L \) such that \( |w| \geq k + 1 \). For instance, let \( w = a^{j^2} \). \\
    \\
    Notice that through processing the first \( k + 1 \) symbols in \( w \), some state \( q \)  must repeat by the Pigeonhole principle. Namely, \( \hat{\delta}(q_0, a^{\beta} = q)\) and \( \hat{\delta}(q, a^{\beta}) = q \), where \( \beta \geq 1 \). \\
    \\
    Since \( \mathcal{D} \) accepts \( w = a^{j^2} \), it follows that \( \hat{\delta}(q_0, a^{j^2}) \in F \). Notice that \( \hat{\delta}(q_0, a^{j^2 + \beta}) = \hat{\delta}(q_0, a^{\alpha + \beta + \gamma + \beta}) = \hat{\delta}(\hat{\delta}(q_0, a^{\alpha}), a^{\beta + \beta + \gamma}) = \hat{\delta}(q, a^{\beta + \beta + \gamma}) = \hat{\delta}(\hat{\delta}(q, a^{\beta}), a^{\beta + \gamma}) = \hat{\delta}(q, a^{\beta + \gamma}) = \hat{\delta}(\hat{\delta}(q_0, a^{alpha}), a^{\beta + \gamma}) = \hat{\delta}(q_0, a^{\alpha + \beta + \gamma}) \in F \). \\
    \\
    This equivalence shows that \( \mathcal{D} \) accepts \( w = a^{j^2 + \beta} \). By the same argument, \( \mathcal{D}  \) also accepts \( a^{j^2 + 2\beta}, a^{j^2 + 3\beta} \). \\
    \\
    However, notice that one of \( a^{j^2 + 2\beta}, a^{j^2 + 3\beta} \) is not a square, which is a contradiction. Therefore, \( L = \{ a^{n^2} \mid n \in \mathbb{N} \} \) must not be regular. \\
\end{proof}
\leavevmode\\
\textit{Here's an alternative proof by using the \textbf{Pumping Lemma}.}
\begin{proof}
\leavevmode\\
    To show that \( L = \{ a^{n^2} \mid n \in \mathbb{N} \} \) is not a regular language, show:
    \[
        (\forall n \in \mathbb{N}^+)(\exists w = xyz \in L)(\exists i \in \mathbb{N}^+)(|y| \geq 1 \land |xy| \leq n + 1 \land xy^iz \notin L) \text{.}
    \]
    Let \( n \in \mathbb{N}^+ \) be arbitrary. \\
    Notice that \( |w| \geq n \). Choose arbitrary \( x, y \in \Sigma^* \) where \( xy \) represents the start (\( \leq n + 1 \) ) of \( w \). This makes \( y \) the part in which state-looping occurs. \\
    \\
    Let \( n \in \mathbb{N}^+ \). Choose \( w = a^{b^2} \), where \( b^2 > n \). \\
    \\
    Now, write \( w = xyz \). \\
    \\
    Notice that \( y = a^i \), where \( 1 \leq  i \leq  n + 1 \). \\
    \\
    \textbf{\textit{Add some proof here to show that...}} \\
    \\
    It follows that either \( xyy^2 = a^{n^2 + i} \notin L \) or \( xyyy^2 = a^{n^2 + 2i} \notin L \). \\
    \\
    Therefore, \( L = \{ a^{n^2} \mid n \in \mathbb{N} \} \) is not a regular language. \\
\end{proof}
\pagebreak

\section*{Question \#4}
Prove that \( L = \{ 0^n1^n \mid n \in \mathbb{N} \} \) is \textbf{not} a regular language.\begin{proof}
\leavevmode\\
    Let \( n \in \mathbb{N} \). Choose \( w = 0^{n + 300}1^{n + 300} \) \\
    \\
    Only case: \( y = 0 \). \\
    \\
    Then, \( w = 0^{n + 300}1^{n + 300} \notin L \). \\
\end{proof}
\pagebreak



\end{document}
