\documentclass[12pt]{article}

\usepackage[utf8]{inputenc}
\usepackage{newunicodechar}
\newunicodechar{ℝ}{\mathbb{R}}
\usepackage{EngReport}
\usepackage{listings}
\usepackage{cancel}
\usepackage{comment}
\usepackage{amssymb}
\usepackage{amsthm}
\usepackage{amsmath}
\usepackage{graphicx}
\usepackage{setspace}
\usepackage{geometry}
\usepackage{xcolor}  % Required for coloring in listings

\graphicspath{{Images/}}
\onehalfspacing
\geometry{letterpaper, portrait, includeheadfoot=true, hmargin=1in, vmargin=1in}

% Define custom colors
\definecolor{myblue}{RGB}{0, 128, 255}
\definecolor{mygreen}{RGB}{34, 139, 34}
\definecolor{myorange}{RGB}{255, 140, 0}
\definecolor{mygray}{RGB}{128, 128, 128}
\definecolor{mypurple}{RGB}{148, 0, 211}
\definecolor{myred}{RGB}{255, 69, 0}

% Configure listings for Python with custom styles
\lstset{
    language=Python,             % Set language to Python
    basicstyle=\ttfamily\small,  % Use a smaller monospace font
    keywordstyle=\color{myblue}\bfseries,  % Keywords in blue and bold
    commentstyle=\color{mygreen}\itshape,  % Comments in green and italic
    stringstyle=\color{myorange},          % Strings in orange
    numberstyle=\color{mygray},            % Line numbers in gray
    identifierstyle=\color{mypurple},      % Functions and variables in purple
    morekeywords={print, len, range},      % Define additional Python keywords
    showstringspaces=false,                % Do not show spaces in strings
    breaklines=true,                       % Enable line breaking
    numbers=left,                          % Add line numbers to the left
    numbersep=5pt,                         % Space between line numbers and code
    frame=single,                          % Add a box around the code
    rulecolor=\color{mygray},              % Frame color
    moredelim=[is][\color{myred}]{@@}{@@}, % Custom inline LaTeX coloring
}

\begin{document}
\renewcommand{\familydefault}{\rmdefault}

\begin{titlepage}
    \null % This is a TeX command that does nothing but is necessary for vfill to work correctly
    \vfill
    \begin{center}
        {\fontsize{40}{48}\selectfont \bfseries CSC236 Exam Review}
        \vspace{20pt} \\
        {\LARGE Notes from CSC236 Lecture 12} \\
        \vspace{20pt}
        \textbf{Alexander He Meng}
        \vspace{8pt}
        \\ Typed on November 27, 2024
    \end{center}
    \vfill
\end{titlepage}

\pagestyle{fancy}
\fancyhf{}
\setlength{\headheight}{30pt}
\renewcommand{\headrulewidth}{0.4pt}
\renewcommand{\footrulewidth}{0.4pt}
\lhead{\large \textbf{CSC236 UTM} \\ \textbf{Mock Exam }\scriptsize(thx ethan!)\normalsize \textbf{ Solutions}}
\rhead{\large \textbf{Fall 2024} \\ \textbf{Prepared for Dec 17}}
\rfoot{\textbf{Page \thepage}}
\lfoot{}
\pagebreak
\normalsize

\section*{Question \#1}
Let \( \mathcal{F} \) be the collection of all functions with domain \( \mathbb{N} \) and co-domain \( \mathbb{R} \). \\
Given \( A, B \in \mathcal{P}(\mathcal{F}) \), define addition on \( \mathcal{P}(\mathcal{F}) \) by
\[
    A + B \coloneq \{ f + g: f \in A, g \in B \} \text{.}
\]
Recall that \( f + g \) is the function with domain \( \mathbb{N} \) and co-domain \( \mathbb{R} \) such that
\[
    (f + g)(n) = f(n) + g(n) \text{.}
\]
Also recall that, if \( h \in \mathcal{F} \), then
\[
    O(h) = \{ q \in \mathcal{F}: (\exists n_0, c \in \mathbb{N})(\forall n \geq n_0)[|q(n)| \leq c|h(n)|] \} \text{.}
\]
\textbf{\underline{Claim:}} For arbitrary nonnegative \( u, v \in \mathcal{F} \), it follows that \( O(u) + O(v) = O(u + v) \).
\begin{proof}
\leavevmode\\
    This proof demonstrates a double-subset inclusion to show equality. \\
    \\
    \underline{Forward Inclusion — \(O(u) + O(v) \subseteq O(u + v)\):} \\
    Let \( h \in [O(u) + O(v)] \). Then, \( h = f + g \), where \( f \in O(u) \) and \( g \in O(v) \). \\
    \\
    By definition, there exists \( c_1, c_2, n_1, n_2 > 0 \) such that
    \begin{itemize}
        \item \( |f(n)| \leq c_1|u(n)| \text{ for all } n \geq n_1 \);
        \item \( |g(n)| \leq c_2|v(n)| \text{ for all } n \geq n_2 \).
    \end{itemize}
    \leavevmode\\
    Choose \( n_0 = max(n_1, n_2) \) and \( c = max(c_1, c_2) \). \\
    Using the definition, triangle inequality, and assumption that \( u, v \) are nonnegative functions, it follows that
    \begin{equation*}
        \begin{aligned}
            |h(n)| &= |f(n) + g(n)| \leq |f(n)| + |g(n)| \leq c_1|u(n)| + c_2|v(n)| \\
            &\leq c|u(n)| + c|v(n)| \\
            &\leq c(|u(n)| + |v(n)|) \\
            &= c(u(n) + v(n)) = c(|u(n) + v(n)|) = c(|(u + v)(n)|) \text{.}
        \end{aligned}
    \end{equation*}
    Thus, \( |h(n)| \leq c|(u + v)(n)| \). \\
    \\
    By definition, \( h \in O(u + v) \). Therefore, \( O(u) + O(v) \subseteq O(u + v) \). \\
    \\
    \underline{Backward Inclusion — \(O(u) + O(v) \supseteq O(u + v)\):} \\
    Let \( h \in O(u + v) \). \\
    \\
    By definition, there exists \( c, n_0 > 0 \) such that \( |h(n)| \leq c|(u + v)(n)| \) for all \( n \geq n_0 \). \\
    \\
    It follows that \( |h(n)| \leq c|u(n) + v(n)| = c(u(n) + v(n)) \), as \( u, v\) are nonnegative functions. \\
    \\
    Let \( w(n) = h(n) - cu(n) \). Consider the following cases for \( w(n) \). \\
    \\
    \underline{Case — \( w(n) > 0 \):} \\
    Notice that \( w(n) > 0 \implies h(n) - cu(n) > 0 \implies h(n) > cu(n) \). \\
    Since \( u \) is a nonnegative function, then \( h(n) \) must be positive. \\
    \\
    Recall \( |h(n)| \leq c|(u + v)(n)| = c|u(n) + v(n)| \), and both \( u, v \) are nonnegative functions. \\
    \\
    It follows that
    \begin{equation*}
        \begin{aligned}
            |w(n)| &= |h(n) - cu(n)| = h(n) - cu(n) \\
            &= |h(n)| - cu(n) \leq |cu(n) + cv(n)| - cu(n) \\
            &= \cancel{cu(n)} + cv(n) - \cancel{cu(n)} = cv(n) = c|v(n)| \text{.}
        \end{aligned}
    \end{equation*}
    Thus, \( |w(n)| \leq c|v(n)| \). This means \( w(n) \in O(v) \). \\
    \\
    Write \( h(n) = cu(n) + w(n) \). It is obvious that \( cu(n) \in O(u) \), and recall that \( w(n) \in O(v) \). \\
    \\
    Thus, \( h(n) \in O(u + v) \). \\
    \\
    \underline{Case — \( w(n) \leq 0 \):} \\
    Notice that \( w(n) \leq 0 \implies h(n) - cu(n) \leq 0 \implies h(n) \leq cu(n) \). So, choose \( h(n) = cu(n) \). \\
    Then, \( w(n) = h(n) - cu(n) = \cancel{cu(n)} - \cancel{cu(n)} = 0 \). \\
    \\
    Notice that \( |w(n)| = |0| = 0 \leq c|v(n)| \), in fact, for any \( c > 0 \). \\
    Clearly, \( w(n) \in O(v) \). \\
    \\
    Write \( h(n) = cu(n) + w(n) \). It is obvious that \( cu(n) \in O(u) \), and recall that \( w(n) \in O(v) \). \\
    \\
    Thus, \( h(n) \in O(u + v) \). \\
    \\
    \underline{Conclusion of Cases:} \\
    In all cases, \( h(n) \in O(u + v) \) has been demonstrated. \\
    \\
    Therefore, \( O(u) + O(v) \subseteq O(u + v) \). \\
    \\
    \underline{Conclusion:} \\
    Since both inclusions hold, \( O(u) + O(v) = O(u + v) \). \\
\end{proof}
\pagebreak

\section*{Question \#2}
Let \( \mathcal{F} \) be as in \textit{Question \#1}. Let \( \mathcal{G} \) be the collection of all functions with domain \( \mathcal{N} \times \mathcal{N} \) and co-domain \( \mathcal{R} \). Let \( V \in \mathcal{G} \). \\
For every \( i \in \mathbb{N} \), let \( g_i(n) = \sum_{j=1}^{i} V(j, n) \), and let \( f_i(n) = V(i, n) \).
\subsection*{(a)}
\textbf{\underline{Claim:}} For all \( i \in \mathbb{N} \), it follows that \( O(g_i) = \sum_{j=0}^{i} O(f_j) \).
\begin{proof}
\leavevmode\\
    Denote the predicate:
    \[
        P(i) \coloneq O(g_i) = \sum_{j=0}^{i} O(f_j)
    \]
    Proceed using the principle of simple induction over \( P(i) \) for all \( i \in \mathbb{N} \). \\
    \\
    \underline{Base Case:} \\
    Let \( i = 0 \). \\
    \\
    Then,
    \begin{equation*}
        \begin{aligned}
            O(g_i) &= O(g_0) \\
            &= O(\sum_{j=0}^{0} V(j, n)) \\
            &= O(V(0, n)) \\
            &= O(f_0) \\
            &= \sum_{j=0}^{0} O(f_j) \\
            &= \sum_{j=0}^{i} O(f_j) \text{.}
        \end{aligned}
    \end{equation*}
    Thus, \( P(0) \). \\
    \\
    \underline{Induction Hypothesis:} \\
    Assume for some \( k \in \mathbb{N} \), \( P(k) \). \\
    \\
    This means \( O(g_k) = \sum_{j=0}^{k} O(f_j) \). \\
    \\
    \underline{Induction Step:} \\
    Notice that
    \begin{equation*}
        \begin{aligned}
            O(g_{k + 1}) &= O(\sum_{j=0}^{k + 1} V(j, n)) \\
            &= O(\sum_{j=0}^{k + 1} f_j) \\
            &= O(\sum_{j=0}^{k} f_j + f_{k + 1}) \\
            &= O(\sum_{j=0}^{k} f_j) + O(f_{k + 1}) \text{, by \textit{Question \#1}} \\
            &= \sum_{j=0}^{k} O(f_j) + O(f_{k + 1}) \text{, by the Induction Hypothesis} \\
            &= \sum_{j=0}^{k + 1} O(f_j) \text{.}
        \end{aligned}
    \end{equation*}
    Thus, \( P(k) \implies P(k + 1) \). \\
    \\
    \underline{Conclusion:} \\
    Therefore, by the principle of simple induction, \( P(i) \) holds for all \( i \in \mathbb{N} \). \\
\end{proof}
\subsection*{(b)}
\textbf{\underline{Claim:}} If \( g(n) = g_n(n) \), then \( O(g) = \sum_{j=0}^{n} O(f_j) \) does \textbf{not} necessarily hold.
\subsection*{(b)}
\textbf{\underline{Claim:}} If \( g(n) = g_n(n) \), then \( O(g) = \sum_{j=0}^{n} O(f_j) \) does \textbf{not} necessarily hold.
\begin{proof}
\leavevmode\\
    To show that the equivalence in the claim does not necessarily hold, consider a counterexample. \\
    \\
    Fix \( n_0 \). Define \( f_j(n) \) as follows:
    \[
    f_j(n) = \begin{cases} 
        n^2 & \text{if } j = n, \\
        1   & \text{if } j \neq n.
    \end{cases}
    \]
    Consider the function \( g_n(n) = \sum_{j=0}^n f_j(n) \).
    \\
    For \( n > n_0 \), compute \( g_n(n) \) as follows:
    \[
    g_n(n) = \sum_{j=0}^{n} f_j(n) = \sum_{j=0}^{n-1} f_j(n) + f_n(n).
    \]
    Substituting the definition of \( f_j(n) \), this leads to:
    \[
    \sum_{j=0}^{n-1} f_j(n) = \sum_{j=0}^{n-1} 1 = n.
    \]
    Since \( f_n(n) = n^2 \), it follows that:
    \[
    g_n(n) = n + n^2.
    \]
    Therefore, \( O(g_n) = O(n + n^2) = O(n^2) \). Let this be the left-hand side (LHS). \\
    \\
    On the other hand, consider \( \sum_{j=0}^n O(f_j) \):
    \[
    f_j(n) = 1 \text{ for all } j \neq n.
    \]
    Hence, \( O(f_j) = O(1) \). There are \( n \) terms where \( f_j(n) = 1 \), so:
    \[
    \sum_{j=0}^n O(f_j) = \sum_{j=0}^n O(1) = (n + 1) O(1).
    \]
    This simplifies to \( O(n + 1) = O(n) \). Let this be the right-hand side (RHS). \\
    \\
    Clearly, \( LHS = O(n^2) \neq O(n) = RHS \). \\
    \\
    Note that this analysis holds for \( n > n_0 \), as \( n_0 \) is fixed and \( n \) can grow arbitrarily large. Fixing \( n_0 \) ensures a concrete starting point, while allowing \( n > n_0 \) provides generality for the counterexample. The counterexample demonstrates that \( O(g) = \sum_{j=0}^{n} O(f_j) \) does not necessarily hold in general. \\
    \\
    Thus, the equivalence in the claim is disproved. \\
\end{proof}
\pagebreak

\section*{Question \#3}
\textbf{\underline{Claim:}} \( f(n) = \lceil \sqrt{n} \rceil - \lfloor \sqrt{n} - 4 \rfloor \) is asymptotically constant (i.e. \( f(n) \in \Theta(1) \)).
\begin{proof}
\leavevmode\\
    By definition, if \(x\) and \(y\) are arbitrary real numbers, then \[ (x \leq \lceil x \rceil < x + 1) \] and \[ (y - 1 < \lfloor y \rfloor \leq y) \text{.} \] \\
    Rewrite the second inequality as \( -y \leq - \lfloor y \rfloor < - (y - 1) \). \\
    \\
    By adding the two inequalities, it follows that \( x - y \leq \lceil x \rceil - \lfloor y \rfloor < x + 1 - (y - 1) = x - y + 2 \). \\
    \\
    Let \( x = \sqrt{n} \) and \( y = \sqrt{n} - 4 \), for arbitrary natural \( n \). \\
    Then, \( \lceil x \rceil - \lfloor y \rfloor = \lceil \sqrt{n} \rceil - \lfloor \sqrt{n} - 4 \rfloor = f(n) \). As well, \( x - y = \cancel{\sqrt{n}} - (\cancel{\sqrt{n}} - 4) = 4 \). \\
    \\
    This means \( x - y \leq \lceil x \rceil - \lfloor y \rfloor < x - y + 2 \implies 4 \leq f(n) < 4 + 2 \implies 4 \leq f(n) < 6 \). \\
    \\
    Let \( n_0 = 0, c = 4, d = 6 \). Let \( g(n) = 1 \). \\
    Notice that \( 4 \leq f(n) < 6 \implies cg(n) \leq f(n) \leq dg(n) \), for all \( n \geq n_0 = 0 \) with \( c = 4, d = 6 \). \\
    \\
    Therefore, \( f(n) \in \Theta(g(n)) \implies f(n) \in \Theta(1) \). Indeed, \( f(n) \) is asymptotically constant. \\
\end{proof}
\pagebreak

\section*{Question \#4}
\textbf{\underline{Claim:}} The recurrence, \( T(n) = 3T(\frac{n}{3}) + n^2 - n \), can be solved using the master theorem, and there exists a function \( g(n) \) such that \( T \in \Theta(g(n)) \).
\begin{proof}
\leavevmode\\
    The recurrence \( T(n) = 3T(\frac{n}{3}) + n^2 - n \) has the form \( T(n) = aT(\frac{n}{b}) + f(n) \), where \( a = 3 \), \( b = 3 \), and \( f(n) = n^2 - n \). Since \( f(n) = n^2 - n \) asymptotically behaves like \( n^2 \), it follows that \( f(n) \in \Theta(n^2) \), implying \( k = 2 \). \\
    \\
    Master theorem applies to recurrences of this form, provided \( a > 0 \), \( b > 1 \), and \( f(n) \) is non-negative for sufficiently large \( n \). Here, \( a = 3 \), \( b = 3 \), and \( f(n) = n^2 - n \) satisfies all these conditions since \( n^2 \) dominates \( n \) as \( n \to \infty \). \\
    \\
    Next, compute \( \log_b a \):
    \[
    \log_b a = \log_3 3 = 1 \text{.}
    \]
    Compare \( \log_b a \) with \( k \):
    \[
    k = 2 > \log_3 3 = 1 \text{.}
    \]
    By the master theorem, when \( k > \log_b a \), this leads to \( T(n) \in \Theta(n^k) \). \\
    Thus:
    \[
    T(n) \in \Theta(n^2) \text{.}
    \]
    \\
    Therefore, there exists a function \( g(n) = n^2 \) such that \( T(n) \in \Theta(g(n)) \). \\
\end{proof}
\pagebreak

\section*{Question \#5}
\textbf{\underline{Claim:}} Every regex without the Kleene star \( * \) represents a finite language.
\begin{proof}
\leavevmode\\
    Let \( r \) be a regular expression without the Kleene star \( * \). \\
    Define the predicate:
    \[
        P(r) \coloneq \mathcal{L}(r) \text{ is a finite language.}
    \]
    Proceed using the principle of structural induction over \( P(r) \) for all regular expressions \( r \) without the Kleene star \( *\). \\
    \\
    \underline{Base Case:} \\
    By the definition of regular expressions,
    \begin{itemize}
        \item \( \mathcal{L}(\varnothing) = \varnothing \)
        \item \( \mathcal{L}(\epsilon) = \{ \epsilon \} \)
        \item \( \mathcal{L}(a) = \{ a \} \), where \( a \in \Sigma \) is an arbitrary symbol 
    \end{itemize}
    Clearly, all three languages as denoted above are finite. \\
    \\
    Thus, \( P(\varnothing), P(\epsilon), P(a) \) all hold. \\
    \\
    \underline{Induction Hypothesis:} \\
    Assume that for some regular expressions \( r_1, r_2 \) without the Kleene star \( * \), \( P(r_1), P(r_2) \) hold. \\
    \\
    This means the languages \( \mathcal{L}(r_1), \mathcal{L}(r_2) \) are finite. \\
    \\
    \underline{Induction Step:} \\
    Consider that every language without the Kleene star \( * \) can be obtained by the union or concatenation of languages. \\
    \\
    By the Induction Hypothesis, \( \mathcal{L}(r_1) \) and \( \mathcal{L}(r_2) \) are finite languages. Recall that languages are sets of elements, where the elements are symbols of some alphabet \( \Sigma \). \\
    \\
    By definition, \( \mathcal{L}(r_1 + r_2) = \mathcal{L}(r_1) \cup \mathcal{L}(r_2) \), which is finite as the union of finite sets (languages) is a finite set. \\
    By definition, \( \mathcal{L}(r_1 r_2) = \mathcal{L}(r_1) \overset{\frown}{\ } \mathcal{L}(r_2) \), which is finite as the concatenation of two finite languages remains finite. \\
    \\
    \underline{Conclusion:} \\
    By the principle of structural induction, every regular expression without the Kleene star \( * \) represents a finite language. \\
\end{proof}
\pagebreak

\section*{Question \#6}
\textbf{\underline{Claim:}} The collection of regular languages is closed under complementation (i.e. if \( L \) is a regular langauge on an alphabet \( \Sigma \), then \( \Sigma^* \setminus L \) is also a regular language).
\begin{proof}
\leavevmode\\
    Suppose \( L \) is a regular langauge over an arbitrary alphabet \( \Sigma \). \\
    Then, there exists a DFA \( \mathcal{D} = (\Sigma, Q, \delta, s, F) \) that accepts \( L \). \\
    \\
    Consider the DFA \( \mathcal{D}' = (\Sigma, Q, \delta, s, Q \setminus F) \), and proceed to show that \( D' \) accepts \( \Sigma^* \setminus L \). \\
    \\
    Let \( w \in \Sigma^* \setminus L \). \\
    Then, \( \mathcal{D} \) rejects \( w \), and \( \delta(s, w) \notin F \implies \delta(s, w) \in Q \setminus F \). Clearly, \( \mathcal{D}' \) accepts \( w \). \\
    \\
    Conversely, let \( x \in \Sigma^* \) such that \( \mathcal{D}' \) accepts \( x \). \\
    Then, \( \delta(s, x) \in Q \setminus F \implies \delta(s, x) \notin F \). This means \( \mathcal{D} \) rejects \( x \), so \( x \notin L \) but \( x \in \Sigma^* \setminus L \). \\
    \\
    Therefore, the DFA \( \mathcal{D}' \) accepts \( \Sigma^* \setminus L \), demonstrating that \( \Sigma^* \setminus L \) is a regular language. \\
\end{proof}
\pagebreak

\section*{Question \#7}
wordsgohere

\begin{proof}
\leavevmode\\
    proofgoeshere \\
\end{proof}
\pagebreak

\end{document}
