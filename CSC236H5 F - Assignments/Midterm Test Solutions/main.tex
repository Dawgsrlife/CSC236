\documentclass[12pt]{article}

\usepackage[utf8]{inputenc}
\usepackage{newunicodechar}
\newunicodechar{ℝ}{\mathbb{R}}
\usepackage{EngReport}
\usepackage{listings}
\usepackage{cancel}
\usepackage{comment}
\usepackage{amssymb}
\usepackage{amsthm}
\usepackage{amsmath}
\usepackage{graphicx}
\usepackage{setspace}
\usepackage{geometry}
\usepackage{xcolor}  % Required for coloring in listings

\graphicspath{{Images/}}
\onehalfspacing
\geometry{letterpaper, portrait, includeheadfoot=true, hmargin=1in, vmargin=1in}

% Define custom colors
\definecolor{myblue}{RGB}{0, 128, 255}
\definecolor{mygreen}{RGB}{34, 139, 34}
\definecolor{myorange}{RGB}{255, 140, 0}
\definecolor{mygray}{RGB}{128, 128, 128}
\definecolor{mypurple}{RGB}{148, 0, 211}
\definecolor{myred}{RGB}{255, 69, 0}

% Configure listings for Python with custom styles
\lstset{
    language=Python,             % Set language to Python
    basicstyle=\ttfamily\small,  % Use a smaller monospace font
    keywordstyle=\color{myblue}\bfseries,  % Keywords in blue and bold
    commentstyle=\color{mygreen}\itshape,  % Comments in green and italic
    stringstyle=\color{myorange},          % Strings in orange
    numberstyle=\color{mygray},            % Line numbers in gray
    identifierstyle=\color{mypurple},      % Functions and variables in purple
    morekeywords={print, len, range},      % Define additional Python keywords
    showstringspaces=false,                % Do not show spaces in strings
    breaklines=true,                       % Enable line breaking
    numbers=left,                          % Add line numbers to the left
    numbersep=5pt,                         % Space between line numbers and code
    frame=single,                          % Add a box around the code
    rulecolor=\color{mygray},              % Frame color
    moredelim=[is][\color{myred}]{@@}{@@}, % Custom inline LaTeX coloring
}

\begin{document}
\renewcommand{\familydefault}{\rmdefault}

\begin{titlepage}
    \null % This is a TeX command that does nothing but is necessary for vfill to work correctly
    \vfill
    \begin{center}
        {\fontsize{40}{48}\selectfont \bfseries CSC236 Exam Review}
        \vspace{20pt} \\
        {\LARGE Notes from CSC236 Lecture 12} \\
        \vspace{20pt}
        \textbf{Alexander He Meng}
        \vspace{8pt}
        \\ Typed on November 27, 2024
    \end{center}
    \vfill
\end{titlepage}

\pagestyle{fancy}
\fancyhf{}
\setlength{\headheight}{30pt}
\renewcommand{\headrulewidth}{0.4pt}
\renewcommand{\footrulewidth}{0.4pt}
\lhead{\large \textbf{CSC236 UTM} \\ \textbf{Mock Exam }\scriptsize(thx ethan!)\normalsize \textbf{ Solutions}}
\rhead{\large \textbf{Fall 2024} \\ \textbf{Prepared for Dec 17}}
\rfoot{\textbf{Page \thepage}}
\lfoot{}
\pagebreak
\normalsize

\section*{Question \#1}
Let \( \mathcal{F} \) be the collection of all functions with domain \( \mathbb{N} \) and co-domain \( \mathbb{R} \). \\
Given \( A, B \in \mathcal{P}(\mathcal{F}) \), define addition on \( \mathcal{P}(\mathcal{F}) \) by
\[
    A + B \coloneq \{ f + g: f \in A, g \in B \} \text{.}
\]
Recall that \( f + g \) is the function with domain \( \mathbb{N} \) and co-domain \( \mathbb{R} \) such that
\[
    (f + g)(n) = f(n) + g(n) \text{.}
\]
Also recall that, if \( h \in \mathcal{F} \), then
\[
    O(h) = \{ q \in \mathcal{F}: (\exists n_0, c \in \mathbb{N})(\forall n \geq n_0)[|q(n)| \leq c|h(n)|] \} \text{.}
\]
\textbf{\underline{Claim:}} For arbitrary nonnegative \( u, v \in \mathcal{F} \), it follows that \( O(u) + O(v) = O(u + v) \).
\begin{proof}
\leavevmode\\
    This proof demonstrates a double-subset inclusion to show equality. \\
    \\
    \underline{Forward Inclusion — \(O(u) + O(v) \subseteq O(u + v)\):} \\
    Let \( h \in [O(u) + O(v)] \). Then, \( h = f + g \), where \( f \in O(u) \) and \( g \in O(v) \). \\
    \\
    By definition, there exists \( c_1, c_2, n_1, n_2 > 0 \) such that
    \begin{itemize}
        \item \( |f(n)| \leq c_1|u(n)| \text{ for all } n \geq n_1 \);
        \item \( |g(n)| \leq c_2|v(n)| \text{ for all } n \geq n_2 \).
    \end{itemize}
    \leavevmode\\
    Choose \( n_0 = max(n_1, n_2) \) and \( c = max(c_1, c_2) \). \\
    Using the definition, triangle inequality, and assumption that \( u, v \) are nonnegative functions, it follows that
    \begin{equation*}
        \begin{aligned}
            |h(n)| &= |f(n) + g(n)| \leq |f(n)| + |g(n)| \leq c_1|u(n)| + c_2|v(n)| \\
            &\leq c|u(n)| + c|v(n)| \\
            &\leq c(|u(n)| + |v(n)|) \\
            &= c(u(n) + v(n)) = c(|u(n) + v(n)|) = c(|(u + v)(n)|) \text{.}
        \end{aligned}
    \end{equation*}
    Thus, \( |h(n)| \leq c|(u + v)(n)| \). \\
    \\
    By definition, \( h \in O(u + v) \). Therefore, \( O(u) + O(v) \subseteq O(u + v) \). \\
    \\
    \underline{Backward Inclusion — \(O(u) + O(v) \supseteq O(u + v)\):} \\
    Let \( h \in O(u + v) \). \\
    \\
    By definition, there exists \( c, n_0 > 0 \) such that \( |h(n)| \leq c|(u + v)(n)| \) for all \( n \geq n_0 \). \\
    \\
    It follows that \( |h(n)| \leq c|u(n) + v(n)| = c(u(n) + v(n)) \), as \( u, v\) are nonnegative functions. \\
    \\
    Let \( w(n) = h(n) - cu(n) \). Consider the following cases for \( w(n) \). \\
    \\
    \underline{Case — \( w(n) > 0 \):} \\
    Notice that \( w(n) > 0 \implies h(n) - cu(n) > 0 \implies h(n) > cu(n) \). \\
    Since \( u \) is a nonnegative function, then \( h(n) \) must be positive. \\
    \\
    Recall \( |h(n)| \leq c|(u + v)(n)| = c|u(n) + v(n)| \), and both \( u, v \) are nonnegative functions. \\
    \\
    It follows that
    \begin{equation*}
        \begin{aligned}
            |w(n)| &= |h(n) - cu(n)| = h(n) - cu(n) \\
            &= |h(n)| - cu(n) \leq |cu(n) + cv(n)| - cu(n) \\
            &= \cancel{cu(n)} + cv(n) - \cancel{cu(n)} = cv(n) = c|v(n)| \text{.}
        \end{aligned}
    \end{equation*}
    Thus, \( |w(n)| \leq c|v(n)| \). This means \( w(n) \in O(v) \). \\
    \\
    Write \( h(n) = cu(n) + w(n) \). It is obvious that \( cu(n) \in O(u) \), and recall that \( w(n) \in O(v) \). \\
    \\
    Thus, \( h(n) \in O(u + v) \). \\
    \\
    \underline{Case — \( w(n) \leq 0 \):} \\
    Notice that \( w(n) \leq 0 \implies h(n) - cu(n) \leq 0 \implies h(n) \leq cu(n) \). So, choose \( h(n) = cu(n) \). \\
    Then, \( w(n) = h(n) - cu(n) = \cancel{cu(n)} - \cancel{cu(n)} = 0 \). \\
    \\
    Notice that \( |w(n)| = |0| = 0 \leq c|v(n)| \), in fact, for any \( c > 0 \). \\
    Clearly, \( w(n) \in O(v) \). \\
    \\
    Write \( h(n) = cu(n) + w(n) \). It is obvious that \( cu(n) \in O(u) \), and recall that \( w(n) \in O(v) \). \\
    \\
    Thus, \( h(n) \in O(u + v) \). \\
    \\
    \underline{Conclusion of Cases:} \\
    In all cases, \( h(n) \in O(u + v) \) has been demonstrated. \\
    \\
    Therefore, \( O(u) + O(v) \subseteq O(u + v) \). \\
    \\
    \underline{Conclusion:} \\
    Since both inclusions hold, \( O(u) + O(v) = O(u + v) \). \\
\end{proof}
\pagebreak

\section*{Question \#X}
wordsgohere

\begin{proof}
\leavevmode\\
    proofgoeshere \\
\end{proof}
\pagebreak

\end{document}
