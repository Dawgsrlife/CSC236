\documentclass[12pt]{article}

\usepackage[utf8]{inputenc}
\usepackage{newunicodechar}
\newunicodechar{ℝ}{\mathbb{R}}
\usepackage{EngReport}
\usepackage{listings}
\usepackage{cancel}
\usepackage{comment}
\usepackage{amssymb}
\usepackage{amsthm}
\usepackage{amsmath}
\usepackage{graphicx}
\usepackage{setspace}
\usepackage{geometry}
\usepackage{xcolor}  % Required for coloring in listings

\graphicspath{{Images/}}
\onehalfspacing
\geometry{letterpaper, portrait, includeheadfoot=true, hmargin=1in, vmargin=1in}

% Define custom colors
\definecolor{myblue}{RGB}{0, 128, 255}
\definecolor{mygreen}{RGB}{34, 139, 34}
\definecolor{myorange}{RGB}{255, 140, 0}
\definecolor{mygray}{RGB}{128, 128, 128}
\definecolor{mypurple}{RGB}{148, 0, 211}
\definecolor{myred}{RGB}{255, 69, 0}

% Configure listings for Python with custom styles
\lstset{
    language=Python,             % Set language to Python
    basicstyle=\ttfamily\small,  % Use a smaller monospace font
    keywordstyle=\color{myblue}\bfseries,  % Keywords in blue and bold
    commentstyle=\color{mygreen}\itshape,  % Comments in green and italic
    stringstyle=\color{myorange},          % Strings in orange
    numberstyle=\color{mygray},            % Line numbers in gray
    identifierstyle=\color{mypurple},      % Functions and variables in purple
    morekeywords={print, len, range},      % Define additional Python keywords
    showstringspaces=false,                % Do not show spaces in strings
    breaklines=true,                       % Enable line breaking
    numbers=left,                          % Add line numbers to the left
    numbersep=5pt,                         % Space between line numbers and code
    frame=single,                          % Add a box around the code
    rulecolor=\color{mygray},              % Frame color
    moredelim=[is][\color{myred}]{@@}{@@}, % Custom inline LaTeX coloring
}

\begin{document}
\renewcommand{\familydefault}{\rmdefault}

\begin{titlepage}
    \null % This is a TeX command that does nothing but is necessary for vfill to work correctly
    \vfill
    \begin{center}
        {\fontsize{40}{48}\selectfont \bfseries CSC236 Exam Review}
        \vspace{20pt} \\
        {\LARGE Notes from CSC236 Lecture 12} \\
        \vspace{20pt}
        \textbf{Alexander He Meng}
        \vspace{8pt}
        \\ Typed on November 27, 2024
    \end{center}
    \vfill
\end{titlepage}

\pagestyle{fancy}
\fancyhf{}
\setlength{\headheight}{30pt}
\renewcommand{\headrulewidth}{0.4pt}
\renewcommand{\footrulewidth}{0.4pt}
\lhead{\large \textbf{CSC236 UTM} \\ \textbf{Mock Exam }\scriptsize(thx ethan!)\normalsize \textbf{ Solutions}}
\rhead{\large \textbf{Fall 2024} \\ \textbf{Prepared for Dec 17}}
\rfoot{\textbf{Page \thepage}}
\lfoot{}
\pagebreak
\normalsize

\section*{Question \#1}
Consider the following program from pg. 53-54 of the course textbook:
\begin{lstlisting}
def avg(A):
    """
    Pre: A is a non-empty list
    Post: Returns the average of the numbers in A
    """
    sum = 0
    i = 0
    while i < len(A):
        sum += A[i]
        i += 1
    return sum / len(A)

print(avg([1, 2, 3, 4]))  # Example usage
\end{lstlisting}
Denote the predicate:
\[
Q(j): \text{At the beginning of the } j^{\text{th}} \text{ iteration, } \texttt{sum} = \sum_{k=0}^{i-1} A[k].
\]
\textbf{\underline{Claim:}} \\
$\forall j \in \{1, \dots, len(A)\}, Q(j)$
\begin{proof}
\leavevmode\\
    remarksgohere \\
    \\
    \underline{Base Case:} \\
    wordsgohere \\
    \\
    \underline{Induction Hypothesis:} \\
    wordsgohere \\
    \\
    \underline{Induction Step:} \\
    wordsgohere \\
    \\
    conclusiongoeshere \\
\end{proof}

\section*{Question \#2}
Recall \(Q(j)\) from Question \# 1. \\
\\
Denote the following predicate:
\[
    Q'(n): 0 \leq n < len(A) \implies Q(n + 1)
\]
\textbf{\underline{Claim:}} \\
Referencing the previous question, proving $\forall j \in \{ 1, \dots, len(A) \}, Q(j)$ is equivalent to proving $\forall j \in \mathbb{N}, Q'(n)$.

\begin{proof}
\leavevmode\\
    explain why it's equivalent
\end{proof}

\section*{Question \#3}
As follows below, Q6-Q10 respectively represent questions 6 through 10 from pp. 64-66 of the course textbook. \\
\\
\textbf{Q6:} \\
Consider the following code:
\begin{lstlisting}
    def f(x):
        """Pre: x is a natural number"""
        a = x
        y = 10
        while a > 0:
            a -= y
            y -= 1
        return a * y
\end{lstlisting}
\textbf{(a):} Loop Invariant Characterizing \texttt{a} and \texttt{y} \\
wordsgohere \\
\\
\textbf{(b):} Why This Function Fails to Terminate \\
wordsgohere \\
\\
\textbf{Q7:} \\
\textbf{(a)} Consider the recursive program below:
\begin{lstlisting}
    def exp_rec(a, b):
        if b == 0:
            return 1
        else if b mod 2 == 0:
            x = exp_rec(a, b / 2)
            return x * x
        else:
            x = exp_rec(a, (b - 1) / 2)
            return x * x * a
\end{lstlisting}
\underline{Preconditions:} \\
wordsgohere \\
\underline{Postconditions:} \\
wordsgohere \\
Denote the following predicate:
\[P(n): somethinghere\]
\textbf{\underline{Claim:}} expresshowthisiscorrect
\begin{proof}
\leavevmode\\
    wordsgohere \\
\end{proof}
\leavevmode\\
\textbf{(b)} Consider the iterative version of the previous program:
\begin{lstlisting}
    def exp_iter(a, b):
        ans = 1
        mult = a
        exp = b
        while exp > 0:
            if exp mod 2 == 1:
                ans *= mult
            mult = mult * mult
            exp = exp // 2
        return ans
\end{lstlisting}
\underline{Preconditions:} \\
wordsgohere
\underline{Postconditions:} \\
wordsgohere
\\
Denote the following predicate:
\[P(n): somethinghere\]
\textbf{\underline{Claim:}} expresshowthisiscorrect
\begin{proof}
\leavevmode\\
    wordsgohere \\
\end{proof}
\leavevmode\\
\textbf{Q8} \\
Consider the following linear time program:
\begin{lstlisting}
    def majority(A):
        """
        Pre: A is a list with more than half its entries equal to x
        Post: Returns the majority element x
        """
        c = 1
        m = A[0]
        i = 1
        while i <= len(a) - 1:
            if c == 0:
                m = A[i]
                c = 1
            else if A[i] == m:
                c += 1
            else:
                c -= 1
            i += 1
        return m
\end{lstlisting}
Denote the following predicate:
\[P(n): somethinghere\]
\textbf{\underline{Claim:}} expresshowthisiscorrect
\begin{proof}
\leavevmode\\
    wordsgohere \\
\end{proof}
\leavevmode\\
\textbf{Q9} \\
Consider the bubblesort algorithm as follows:
\begin{lstlisting}
    def bubblesort(L):
        """
        Pre: L is a list of numbers
        Post: L is sorted
        """
        k = 0
        while k < len(L):
            i = 0
            while i < len(L) - k - 1:
                if L[i] > L[i + 1]:
                    swap L[i] and L[i + 1]
                i += 1
            k += 1
\end{lstlisting}
\textbf{(a):} Denote the inner loop's invariant:
\[P(n): somethinghere\]
\textbf{\underline{Claim:}} proveinnerloop
\begin{proof}
\leavevmode\\
    wordsgohere \\
\end{proof}
\leavevmode
\textbf{(b):} Denote the outer loop's invariant:
\[P(n): somethinghere\]
\textbf{\underline{Claim:}} proveouterloop
\begin{proof}
\leavevmode\\
    wordsgohere \\
\end{proof}
\leavevmode
\textbf{(c):} Denote the following predicate:
\[P(n): somethinghere\]
\textbf{\underline{Claim:}} expresshowthisiscorrect
\begin{proof}
\leavevmode\\
    wordsgohere \\
\end{proof}
\leavevmode\\
\textbf{Q10} \\
Consider the following generalization of the \texttt{min} function:
\begin{lstlisting}
    def extract(A, k):
        pivot = A[0]
        # Use partition from quicksort
        L, G = partition(A[1, ..., len(A) - 1], pivot)
        if len(L) == k - 1:
            return pivot
        else if len(L) >= k:
            return extract(L, k)
        else:
            return extract(G, k - len(L) - 1)
\end{lstlisting}
\textbf{(a):} Proof of Correctness \\
\[P(n): somethinghere\]
\textbf{\underline{Claim:}} proofofcorrectnessclaim
\begin{proof}
\leavevmode\\
    wordsgohere \\
\end{proof}
\leavevmode\\
\textbf{(b):} Worst-Case Runtime \\
wordsgohere \\
\\

\section*{Question \#4}
As follows below, VI, VII, X, XII, and XIV respectively represent questions 6, 7, 10, 12, and 14 from pp. 46-48 of the course textbook. \\
\\
\textbf{VI} \\
Let $T(n)$ be the number of binary strings of length $n$ in which there are no consecutive $1$'s. So, \(T(0) = 1, T(1) = 2, T(2) = 3, ...\), etc. \\
\\
\textbf{(a):} Recurrence for $T(n)$: \\
recurrencehere \\
\\
\textbf{(b):} Closed Form Expression for $T(n)$: \\
closedformhere \\
\\
\textbf{(c):} Proof of Correctness of Closed Form Expression \\
Denote the following predicate: \\
\[P(n): somethinghere\]
\textbf{\underline{Claim:}} expresshowthisiscorrect
\begin{proof}
\leavevmode\\
    wordsgohere \\
\end{proof}
\leavevmode\\
\textbf{VII} \\
Let $T(n)$ denote the number of distinct full binary trees with $n$ nodes. For example, \(T(1) = 1, T(3) = 1, and T(7) = 5\). Note that every full binary tree has an odd number of nodes. \\
\textbf{\underline{Recurrence for $T(n)$:}} \\
recurrencehere \\
\[P(n): somethinghere\]
\textbf{\underline{Claim:}} \(T(n) \geq (\frac{1}{n})(2)^{(n - 1) / 2}\)
\begin{proof}
\leavevmode\\
    wordsgohere \\
\end{proof}
\leavevmode\\
\textbf{X} \\
A \textit{block} in a binary string is a maximal substring consisting of the same symbol. For example, the string \texttt{0100011} has four blocks: \texttt{0}, \texttt{1}, \texttt{000}, and \texttt{11}. Let $H(n)$ denote the number of binary strings of length $n$ that have no odd length blocks of \texttt{1}'s. For example, $H(4) = 5$:
\[
0000 \;\; 1100 \;\; 0110 \;\; 0011 \;\; 1111
\]
\textbf{\underline{Recursive Function for $H(n)$:}} \\
\[P(n): somethinghere\]
\textbf{\underline{Claim:}} proveouterloop
\begin{proof}
\leavevmode\\
    wordsgohere \\
\end{proof}
\leavevmode\\
\textbf{\underline{Closed Form for $H$ (Using Repeated Substitution):}} \\
\\
\textbf{XII} \\
Consider the following function:
\begin{lstlisting}
    def fast_rec_mult(): # maybe params needed?
    """FILL THIS IN!!!"""
\end{lstlisting}
\textbf{\underline{Worst-Case Runtime Analysis:}} \\
wordsgohere
\\
\textbf{XIV} \\
Recall the recurrence for the worst-case runtime of quicksort:
\[ \begin{dcases}
    c, &\text{ if } n \leq 1; \\
    T(|L|) + T(|G|) + dn, &\text{ if } n > 1.
\end{dcases}
\] where $L$ and $G$ are the partitions of the list. \\
\\
For simplicity, ignore that each list has size \(\frac{n-1}{2}\). \\
\\
\textbf{(a):} Assume the lists are always evenly split; that is, \(|L| = |G| = \frac{n}{2}\) at each recursive call. \\
\textbf{\underline{Tight Asymptotic Bound on the Runtime of Quicksort:}} \\
determinehere \\
\\
\textbf{(b):} Assume the lists are always very unevenly split; that is, \(|L| = n - 2\) and \(|G| = 1\) at each recursive call. \\
\textbf{\underline{Tight Asymptotic Bound on the Runtime of Quicksort:}} \\
determinehere \\
\\
\end{document}
